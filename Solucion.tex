\documentclass{article}
\author{Mateo Bouchet Agudo}
\title{Ejercicios resueltos}

\usepackage[top=2.5cm, bottom=2.5cm, left=3cm, right=3cm]{geometry}
\usepackage{amsmath}
\usepackage{amsfonts}
\usepackage{amssymb}
\usepackage{mathtools}\usepackage{physics}
\usepackage{cancel}
\usepackage{graphicx}
\usepackage{float}
\usepackage{ragged2e}

\newcommand\undermat[2]{%
  \makebox[0pt][l]{$\smash{\underbrace{\phantom{%
    \begin{matrix}#2\end{matrix}}}_{\text{$#1$}}}$}#2}

\begin{document}

\maketitle

\justify

\begin{enumerate}
\textbf{Discute el siguiente sistema en funci\'on del valor del par\'ametro $a$}
$$\left \{\begin{array}{ll}\mathbf{a x + y + z = 1}\\\mathbf{a y + x + z = a}\\\mathbf{x + y + 2 z = a^{2}}\\\end{array}\right.$$
\end{enumerate}
En primer lugar, vamos a obtener la matriz asociada al sistema.
$$\overbrace{\left(\begin{array}{ccc|c}a & 1 & 1 & 1 \\1 & a & 1 & a \\\undermat{A}{1 & 1 & 2} & a^{2} \\\end{array}\right)}^{A*}\rightarrow \begin{array}{cc} \text{donde A es la matriz del sistema} \\\text{y A* la matriz ampliada} \end{array}$$\\

Vamos a estudiar el rango de la matriz A. Para ello, vamos a usar el determinante. Si det(A) = 0, entonces Rg(A) $<$ 3. En otro caso, Rg(A) = 3.
$$\begin{vmatrix}a & 1 & 1 \\1 & a & 1\\1 & 1 & 2\end{vmatrix}=0\to 2 a^{2} - 2 a=0\to\left\{\begin{array}{cc}a_1=0\\a_2=1\end{array}\right.$$

Si a $\neq$ $0$, $1$, entonces Rg(A) = 3. Adem\'as, como A est\'a contenida en A*, Rg(A*) = Rg(A) = n$^{\text{o}}$ inc\'ognitas = 3. Por tanto, seg\'un el teorema de Rouch\'e-Frobenius, el sistema ser\'a compatible determinado y la soluci\'on general la podemos obtener empleando la regla de Cramer.
$$
x=\frac{\begin{vmatrix}1 & 1 & 1 \\a & a & 1\\a^{2} & 1 & 2\end{vmatrix}}{|A|}=\frac{- a^{3} + a^{2} + a - 1}{2 a^{2} - 2 a}=\frac{1 - a^{2}}{2 a}$$$$
y=\frac{\begin{vmatrix}a & 1 & 1 \\1 & a & 1\\1 & a^{2} & 2\end{vmatrix}}{|A|}=\frac{- a^{3} + 3 a^{2} - a - 1}{2 a^{2} - 2 a}=- \frac{a}{2} + 1 + \frac{1}{2 a}$$$$
z=\frac{\begin{vmatrix}a & 1 & 1 \\1 & a & a\\1 & 1 & a^{2}\end{vmatrix}}{|A|}=\frac{a^{4} - 2 a^{2} + 1}{2 a^{2} - 2 a}=\frac{a^{3} + a^{2} - a - 1}{2 a}$$\\

Si a = $0$, la matriz que obtenemos es la siguiente:
$$
\left(\begin{array}{ccc|c}0 & 1 & 1 & 1 \\1 & 0 & 1 & 0\\1 & 1 & 2 & 0\end{array}\right)$$

Nos podemos dar cuenta que el Rg(A) = 2 ya que existe un menor 2x2 que es distinto de 0. Si tomamos el menor en la posici\'on (1, 1):

$$
\begin{vmatrix}0 & 1\\1 & 2\end{vmatrix} = -1 \neq 0$$

En lo referente a A*, existe un menor 3x3 que es distinto de 0. Por ejemplo $$
\begin{vmatrix}0 & 1 & 1 \\1 & 0 & 0\\1 & 1 & 0\end{vmatrix} = 1 \neq 0$$

Esto implica que Rg(A*) = 3 $>$ Rg(A). Por tanto, seg\'un el teorema de Rouch\'e-Frobenius, el sistema es incompatible.\\

Si a = $1$, la matriz que obtenemos es la siguiente:
$$
\left(\begin{array}{ccc|c}1 & 1 & 1 & 1 \\1 & 1 & 1 & 1\\1 & 1 & 2 & 1\end{array}\right)$$

Nos podemos dar cuenta que el Rg(A) = 2 ya que existe un menor 2x2 que es distinto de 0. Si tomamos el menor en la posici\'on (1, 1):

$$
\begin{vmatrix}1 & 1\\1 & 2\end{vmatrix} = 1 \neq 0$$

Adem\'as, el Rg(A*) = Rg(A) = 2 ya que no encontramos ning\'un menor 3x3 distinto de 0 en la matriz ampliada. Empleando el teorema de Rouch\'e-Frobenius, ya que Rg(A) = Rg(A*) $\neq$ n$^{\text{o}}$ inc\'ognitas = 3, concluimos que el sistema es compatible indeterminadode orden 1.

Para encontrar la soluci\'on vamos a hacer Gauss por filas. 
$$
\left(\begin{array}{ccc|c}1 & 1 & 1 & 1 \\1 & 1 & 1 & 1\\1 & 1 & 2 & 1\end{array}\right)\xrightarrow{F_2\to F_2-1F_1}\left(\begin{array}{ccc|c}1 & 1 & 1 & 1 \\0 & 0 & 0 & 0\\1 & 1 & 2 & 1\end{array}\right)\xrightarrow{F_3\to F_3-1F_1}\left(\begin{array}{ccc|c}1 & 1 & 1 & 1 \\0 & 0 & 0 & 0\\0 & 0 & 1 & 0\end{array}\right)$$
$$$$

Con la matriz escalonada es trivial obtener la soluci\'on al sistema: $$
\left\{\begin{array}{lll}x = 1 - \lambda\\ y = \lambda\\ z = 0\end{array}\right.$$\\


\end{document}