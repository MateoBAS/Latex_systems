\documentclass{article}
\author{Mateo Bouchet Agudo}
\title{Ejercicios resueltos}

\usepackage[top=2.5cm, bottom=2.5cm, left=3cm, right=3cm]{geometry}
\usepackage{amsmath}
\usepackage{amsfonts}
\usepackage{amssymb}
\usepackage{mathtools}\usepackage{physics}
\usepackage{cancel}
\usepackage{graphicx}
\usepackage{float}
\usepackage{ragged2e}

\newcommand\undermat[2]{%
  \makebox[0pt][l]{$\smash{\underbrace{\phantom{%
    \begin{matrix}#2\end{matrix}}}_{\text{$#1$}}}$}#2}

\begin{document}

\maketitle

\justify

\begin{enumerate}
\textbf{Discute el siguiente sistema en funci\'on del valor del par\'ametro $a$}
$$\left \{\begin{array}{ll}\mathbf{a x + 2 a y + z = 1}\\\mathbf{a z + 2 x - y = 1}\\\mathbf{a^{2} x + a z - y = a}\\\end{array}\right.$$
\end{enumerate}
En primer lugar, vamos a obtener la matriz asociada al sistema.
$$\overbrace{\left(\begin{array}{ccc|c}a & 2 a & 1 & 1 \\2 & -1 & a & 1 \\\undermat{A}{a^{2} & -1 & a} & a \\\end{array}\right)}^{A*}\rightarrow \begin{array}{cc} \text{donde A es la matriz del sistema} \\\text{y A* la matriz ampliada} \end{array}$$\\

Vamos a estudiar el rango de la matriz A. Para ello, vamos a usar el determinante. Si det(A) = 0, entonces Rg(A) $<$ 3. En otro caso, Rg(A) = 3.
$$\begin{vmatrix}a & 2 a & 1 \\2 & -1 & a\\a^{2} & -1 & a\end{vmatrix}=0\to 2 a^{4} - 3 a^{2} - 2=0\to\left\{\begin{array}{cc}a_1=- \sqrt{2}\\a_2=\sqrt{2}\end{array}\right.$$

Si a $\neq$ $- \sqrt{2}$, $\sqrt{2}$, entonces Rg(A) = 3. Adem\'as, como A est\'a contenida en A*, Rg(A*) = Rg(A) = n$^{\text{o}}$ inc\'ognitas = 3. Por tanto, seg\'un el teorema de Rouch\'e-Frobenius, el sistema ser\'a compatible determinado y la soluci\'on general la podemos obtener empleando la regla de Cramer.
$$
x=\frac{\begin{vmatrix}1 & 2 a & 1 \\1 & -1 & a\\a & -1 & a\end{vmatrix}}{|A|}=\frac{2 a^{3} - 2 a^{2} + a - 1}{2 a^{4} - 3 a^{2} - 2}=\frac{a - 1}{a^{2} - 2}$$$$
y=\frac{\begin{vmatrix}a & 1 & 1 \\2 & 1 & a\\a^{2} & a & a\end{vmatrix}}{|A|}=0=0$$$$
z=\frac{\begin{vmatrix}a & 2 a & 1 \\2 & -1 & 1\\a^{2} & -1 & a\end{vmatrix}}{|A|}=\frac{2 a^{3} - 4 a^{2} + a - 2}{2 a^{4} - 3 a^{2} - 2}=\frac{a - 2}{a^{2} - 2}$$\\

Si a = $- \sqrt{2}$, la matriz que obtenemos es la siguiente:
$$
\left(\begin{array}{ccc|c}- \sqrt{2} & - 2 \sqrt{2} & 1 & 1 \\2 & -1 & - \sqrt{2} & 1\\2 & -1 & - \sqrt{2} & - \sqrt{2}\end{array}\right)$$

Nos podemos dar cuenta que el Rg(A) = 2 ya que existe un menor 2x2 que es distinto de 0. Si tomamos el menor en la posici\'on (2, 1):

$$
\begin{vmatrix}- 2 \sqrt{2} & 1\\-1 & - \sqrt{2}\end{vmatrix} = 5 \neq 0$$

En lo referente a A*, existe un menor 3x3 que es distinto de 0. Por ejemplo $$
\begin{vmatrix}- \sqrt{2} & - 2 \sqrt{2} & 1 \\2 & -1 & 1\\2 & -1 & - \sqrt{2}\end{vmatrix} = -10 - 5 \sqrt{2} \neq 0$$

Esto implica que Rg(A*) = 3 $>$ Rg(A). Por tanto, seg\'un el teorema de Rouch\'e-Frobenius, el sistema es incompatible.\\

Si a = $\sqrt{2}$, la matriz que obtenemos es la siguiente:
$$
\left(\begin{array}{ccc|c}\sqrt{2} & 2 \sqrt{2} & 1 & 1 \\2 & -1 & \sqrt{2} & 1\\2 & -1 & \sqrt{2} & \sqrt{2}\end{array}\right)$$

Nos podemos dar cuenta que el Rg(A) = 2 ya que existe un menor 2x2 que es distinto de 0. Si tomamos el menor en la posici\'on (2, 1):

$$
\begin{vmatrix}2 \sqrt{2} & 1\\-1 & \sqrt{2}\end{vmatrix} = 5 \neq 0$$

En lo referente a A*, existe un menor 3x3 que es distinto de 0. Por ejemplo $$
\begin{vmatrix}\sqrt{2} & 2 \sqrt{2} & 1 \\2 & -1 & 1\\2 & -1 & \sqrt{2}\end{vmatrix} = -10 + 5 \sqrt{2} \neq 0$$

Esto implica que Rg(A*) = 3 $>$ Rg(A). Por tanto, seg\'un el teorema de Rouch\'e-Frobenius, el sistema es incompatible.\\


\end{document}